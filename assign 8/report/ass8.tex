	\documentclass[12pt]{book}
\usepackage[a4paper, total={6in, 8in}]{geometry}
\usepackage{amssymb}
\usepackage{listings}
\usepackage{color}
\usepackage{graphicx}
\usepackage{subfig}
\usepackage{float}
\definecolor{mygrey}{gray}{.96} % Light Grey
\definecolor{BrickRed}{RGB}{120,0,0}


\def\xbf{\mathbf{x}}
\def\zbf{\mathbf{z}}
\def\xibf{\mathbf{\xi}}
\lstset{
	language=R,              % choose the language of the code ("language=Verilog" is popular as well)
   tabsize=3,							  % sets the size of the tabs in spaces (1 Tab is replaced with 3 spaces)
	basicstyle=\footnotesize,               % the size of the fonts that are used for the code
	numbers=left,                   % where to put the line-numbers
	numberstyle=\footnotesize,              % the size of the fonts that are used for the line-numbers
	stepnumber=1,                   % the step between two line-numbers. If it's 1 each line will be numbered
	numbersep=5pt,                  % how far the line-numbers are from the code
	backgroundcolor=\color{mygrey}, % choose the background color. You must add \usepackage{color}
	showspaces=false,              % show spaces adding particular underscores
	showstringspaces=false,        % underline spaces within strings
	showtabs=false,                % show tabs within strings adding particular underscores
	frame=single,	                 % adds a frame around the code
	tabsize=3,	                    % sets default tabsize to 2 spaces
	captionpos=b,                   % sets the caption-position to bottom
	breaklines=true,                % sets automatic line breaking
	breakatwhitespace=false,        % sets if automatic breaks should only happen at whitespace
	%escapeinside={\%*}{*)},        % if you want to add a comment within your code
	commentstyle=\color{BrickRed}   % sets the comment style
}

\begin{document}
\title{\textbf{Monte Carlo Simulation Lab Assignment-8}}
\author{Yash Vanjani\\(140123046)\\Mathematics and Computing\\IIT Guwahati}
\date{April 5th, 2016}

\maketitle

\newpage
\begin{enumerate}
\item[Q 1] Use the following Monte Carlo estimator to approximate the expected value \newline $$I=E(exp(\sqrt{U}))$$ \newline where $U\backsim{u}[0,1]:I_M=\frac{1}{M}$ $\sum_{i=1}^M Y_i,$ where $Y_i= exp(\sqrt{U_i}$ with $U_i\backsim u[0,1].$ \newline Take all values of $M$ to be $10^2,10^3,10^4$ and $10^5.$  Determine the $95\%$ confidence interval for $I_M$ for all the four values of $M$ that you have taken\\
\end{enumerate}
\textbf{Solution}: R code:
\begin{lstlisting}
m<-vector("numeric")
m[1]=10^2
m[2]=10^3
m[3]=10^4
m[4]=10^5
for(i in 1:4)
{
	u<-runif(m[i])
	Y<-vector("numeric")
	for(j in 1:m[i])
	{
		Y[j]=exp(sqrt(u[j]))
	}
	I=sum(Y)/m[i]
	var=var(Y)
	min=I-(1.95*sqrt(var)/sqrt(m[i]))
	max=I+(1.95*sqrt(var)/sqrt(m[i]))
	cat("\nI_",m[i]," = ",I,"\n")
	cat("var_",m[i]," = ",var,"\n")
	cat("min_",m[i]," = ",min,"\n")
	cat("max_",m[i]," = ",max,"\n")
}
\end{lstlisting}
\newpage
\textbf{\large Observations :-} \newline
The result is given as the following table with columns M, mean of random variable, variance of random variable and upper and lower endpoint of $95\%$ confidence interval.

\begin{center}
 \begin{tabular}{||c c c c c||} 
 \hline
 M & Mean & Variance & Lower endpoiont & Upper endpoint\\ [0.5ex] 
 \hline\hline
 $10^2$ & $2.013587$ & $0.2133722$ & $1.923512$ & $2.103662$\\ 
 \hline
 $10^3$ & $2.004278$ & $0.1839995$ & $1.977826$ & $2.030729$\\
 \hline
 $10^4$ & $2.005147$ & $0.1936897$ & $1.996565$ & $2.013729$\\
 \hline
 $10^5$ & $1.999698$ & $0.1940106$ & $1.996982$ & $2.002414$\\  [1ex] 
 \hline
\end{tabular}
\end{center}

\textbf{\large Results :-} \newline
1. The theoretical mean is 2. \newline
2. The empirical mean tends to 2 as number of iterations increase. \newline
3. $95\%$ confidence interval for $10^5$ is $( 1.996982,2.002414 )$.

\newpage
\begin{enumerate}
\item[Q 2] Repeat the above exercise using antithetic variates via t
he following estimator and calculate the percentage of variance reduction:
\newline $I_M=\frac{1}{M}$ $\sum_{i=1}^M Y_i$ \newline where $Y_i=\frac{exp(\sqrt{U_i}) + exp (\sqrt{1- {U_i}}}{2} $ \newline with $ U_i \backsim u[0,1] $\\
\end{enumerate}

\textbf{Solution}: R code:
\begin{lstlisting}
m<-vector("numeric")
m[1]=10^2
m[2]=10^3
m[3]=10^4
m[4]=10^5
for(i in 1:4)
{
	u<-runif(m[i])
	Y<-vector("numeric")
	for(j in 1:m[i])
	{
		Y[j]=exp(sqrt(u[j]))
	}
	I_y=sum(Y)/m[i]
	var_y=var(Y)
	Z<-vector("numeric")
	for(j in 1:m[i])
	{
		Z[j]=(exp(sqrt(u[j]))+exp(sqrt(1-u[j])))/2
	}
	I_z=sum(Z)/m[i]
	var_z=var(Z)
	min=I_z-(1.95*sqrt(var_z)/sqrt(m[i]))
	max=I_z+(1.95*sqrt(var_z)/sqrt(m[i]))
	cat("\nI_",m[i]," = ",I_z,"\n")
	cat("var_",m[i]," = ",var_z,"\n")
	cat("min_",m[i]," = ",min,"\n")
	cat("max_",m[i]," = ",max,"\n")
	delta=(var_y-var_z)/var_y*100
	cat("Percentage of variance reduction for ",m[i]," random numbers = ",delta,"\n")
}
\end{lstlisting}

\newpage
\textbf{\large Observations :-} \newline
The result is given as the following table with columns M, mean of random variable, variance of random variable and upper and lower endpoint of $95\%$ confidence interval and variance reduction percentage.

\begin{center}
 \begin{tabular}{||c c c c c c||} 
 \hline
 M & Mean & Variance & Lower endpoiont & Upper endpoint & Variance Reduction ($\%$age)\\ [0.5ex] 
 \hline\hline
 $10^2$ & $1.99814$ & $0.001078075$ & $1.991738$ & $2.004543$ & $99.48891$\\ 
 \hline
 $10^3$ & $1.998315$ & $0.001081175$ & $1.996287$ & $2.000342$ & $99.47121$\\
 \hline
 $10^4$ & $1.999851$ & $0.001102568$ & $1.999204$ & $2.000499$ & $99.4355$\\
 \hline
 $10^5$ & $2.00008$ & $0.001068032$ & $1.999878$ & $2.000281$ & $99.45017$\\  [1ex] 
 \hline
\end{tabular}
\end{center}

\textbf{\large Results :-} \newline
1. The theoretical mean is 2. \newline
2. The empirical mean tends to 2 as number of iterations increase. \newline
3. $95\%$ confidence interval for $10^5$ is $( 1.999878,2.000281 )$. \newline
4. The variance reduction for $10^5$ iterations is $99.45017\%$ which is
desirable.

\newpage

\begin{enumerate}
\item[Q 3] Use $\sqrt{U}$ to construct control variate estimate and repeat the above exercise. Calculate the percentage of variance reduction.
\end{enumerate}

\textbf{Solution}: R code: Using $\sqrt{U}$ as control variate estimate 
\begin{lstlisting}
m<-vector("numeric")
m[1]=10^2
m[2]=10^3
m[3]=10^4
m[4]=10^5
for(i in 1:4)
{
	u<-runif(m[i])
	Y<-vector("numeric")
	for(j in 1:m[i])
	{
		Y[j]=exp(sqrt(u[j]))
	}
	I_y=sum(Y)/m[i]
	var_y=var(Y)

	v<-runif(m[i])
	Z<-vector("numeric")
	for(j in 1:m[i])
	{
		Z[j]=sqrt(u[j])
	}
	I_z=sum(Z)/m[i]
	var_z=var(Z)
	c<-(-1)*cov(Y,Z)/var(Z)

	W<-vector("numeric")
	for(j in 1:m[i])
	{
		W[j]=Y[j]+(c*(Z[j]-I_z))
	}
	I_w=sum(W)/m[i]
	var_w=var(W)

	min=I_z-(1.95*sqrt(var(W))/sqrt(m[i]))
	max=I_z+(1.95*sqrt(var(W))/sqrt(m[i]))

	cat("\nI_",m[i]," = ",I_w,"\n")
	cat("var_",m[i]," = ",var_w,"\n")
	cat("min_",m[i]," = ",min,"\n")
	cat("max_",m[i]," = ",max,"\n")
	delta=(var_y-var_w)/var_y*100
	cat("Percentage of variance reduction for ",m[i]," random numbers = ",delta,"\n")
}
\end{lstlisting}
\textbf{\large Observations :-} \newline
\textbf {Using $\sqrt{U}$ as control variate estimate} \newline
The result is given as the following table with columns M,mean of random variable, variance of random variable and upper and lower endpoint of $95\%$ confidence interval and variance reduction percentage

\begin{center}
 \begin{tabular}{||c c c c c c||} 
 \hline
 M & Mean & Variance & Lower endpoiont & Upper endpoint & Variance Reduction ($\%$age)\\ [0.5ex] 
 \hline\hline
 $10^2$ & $1.877849$ & $0.002499348$ & $0.5927332$ & $0.6122307$ & $98.70001$\\ 
 \hline
 $10^3$ & $1.97944$ & $0.002635716$ & $0.6537804$ & $0.660112$ & $98.59785$\\
 \hline
 $10^4$ & $2.004715$ & $0.002702267$ & $0.6682152$ & $0.6702425$ & $98.60314$\\
 \hline
 $10^5$ & $2.003103$ & $0.002697464$ & $0.6679026$ & $0.6685431$ & $98.61584$\\  [1ex] 
 \hline
\end{tabular}
\end{center}
\textbf{\large Results :-} \newline
1. The theoretical mean is 2. \newline
2. The empirical mean tends to 2 as number of iterations increase. \newline
3. $95\%$ confidence interval for $10^5$ using $\sqrt{U}$ is $( 0.6679026,0.6685431 )$. \newline
5. The variance reduction is significantly less than antithetic method. \newline

\newpage

\end{document}
